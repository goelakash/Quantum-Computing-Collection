\documentclass[12pt]{article}

\usepackage[T1]{fontenc}
\usepackage[utf8]{inputenc}
\usepackage{graphicx}
\usepackage{xcolor}

\usepackage{tgtermes}

\usepackage[
pdftitle={Quantum Computing -- A Gentle Introduction}, 
pdfauthor={Quick Notes by Aaron Vontell, MIT},
colorlinks=true,linkcolor=blue,urlcolor=blue,citecolor=blue,bookmarks=true,
bookmarksopenlevel=2]{hyperref}
\usepackage{amsmath,amssymb,amsthm,textcomp}
\usepackage{enumerate}
\usepackage{multicol}
\usepackage{tikz}

\usepackage{geometry}
\geometry{total={210mm,297mm},
left=25mm,right=25mm,%
bindingoffset=0mm, top=20mm,bottom=20mm}


\linespread{1.3}

\newcommand{\linia}{\rule{\linewidth}{0.5pt}}

% custom theorems if needed
\newtheoremstyle{mytheor}
    {1ex}{1ex}{\normalfont}{0pt}{\scshape}{.}{1ex}
    {{\thmname{#1 }}{\thmnumber{#2}}{\thmnote{ (#3)}}}

\theoremstyle{mytheor}
\newtheorem{defi}{Definition}

% my own titles
\makeatletter
\renewcommand{\maketitle}{
\begin{center}
\vspace{2ex}
{\huge \textsc{\@title}}
\vspace{1ex}
\\
by Eleanor Rieffel and Wolfgang Polak\\
\linia\\
\@author \hfill \@date
\vspace{4ex}
\end{center}
}
\makeatother
%%%

% custom footers and headers
\usepackage{fancyhdr,lastpage}
\pagestyle{fancy}
\lhead{}
\chead{}
\rhead{}
\lfoot{}
\cfoot{}
\rfoot{Page \thepage\ /\ \pageref*{LastPage}}
\renewcommand{\headrulewidth}{0pt}
\renewcommand{\footrulewidth}{0pt}
%

%%%----------%%%----------%%%----------%%%----------%%%

\begin{document}

\title{Quantum Computing -- A Gentle Introduction}

\author{Quick Notes by Aaron Vontell, MIT EECS}

\date{07/19/2016}

\maketitle

\section{Single-Qubit Quantum Systems}

\subsection{Single Quantum Bits}

\begin{defi}
A \textbf{basis} is a set of vectors for which every element in our space $V$ can be written \textit{uniquely} as a linear combination of these vectors.
\end{defi}
In quantum mechanics, bases are usually required to be orthonormal. Additionally, we define the \textit{standard basis} to be \(\{|0\rangle, |1\rangle\}\).
\\A \textit{ket} \(|v\rangle = a|0\rangle + b|1\rangle\) may be written as $\begin{pmatrix}a\\ b\end{pmatrix}$, while a \textit{bra} \(\langle v|\) can be written as $\begin{pmatrix}a & b\end{pmatrix}$.
\\When we wish to measure a qubit, we mean that we want to measure the qubit with respect to a given basis.
\\
\subsection{The State Space of a Single Qubit System}

\begin{defi}
The \textbf{state space} of a classical or quantum physical system is the set of all possible states of the system, i.e. the set of possible qubit values. Do not be fooled by states that are written differently, but are actually the same state!
\end{defi}

\begin{defi}
The \textbf{global phase} is the multiple by which two vectors represent the same quantum state, and has no physical meaning, where \(|v\rangle = c|v'\rangle\) for some \(c=e^{i\phi}\).
\end{defi}

\begin{defi}
The \textbf{relative phase} of a single qubit system \(a|0\rangle + b|1\rangle\) is a measure of the angle between the two complex numbers $a$ and $b$ in the complex plane. Same magnitude but different relative phase \= different states.
\end{defi}

These states may be visualized in two ways, using either the \textbf{Extended Complex Plane}, or with the \textbf{Bloch Sphere}. See the \textit{bloch\_sphere.png} image in the images folder for an example. Also note that all quantum states can be represented by a wave function, or a solution to the Schr\"odinger Wave Equation.

\textbf{Important single-qubit states}

$$|+\rangle = \frac{1}{\sqrt{2}}(|0\rangle + |1\rangle)$$
$$|-\rangle = \frac{1}{\sqrt{2}}(|0\rangle - |1\rangle)$$
$$|i\rangle = \frac{1}{\sqrt{2}}(|0\rangle + i|1\rangle)$$
$$|-i\rangle = \frac{1}{\sqrt{2}}(|0\rangle - i|1\rangle)$$

\subsection{Chapter Exercises}

\paragraph{Problem 1}
Given a polaroid A with polarization $|v_A\rangle = |\rightarrow\rangle$, B with polarization $|v_B\rangle = \cos\theta|\rightarrow\rangle + \sin\theta|\uparrow\rangle$, and C with polarization $|v_C\rangle = |\uparrow\rangle$, the percentage of photons that make it through $A \rightarrow B \rightarrow C \rightarrow$ can be found in the following way. Choosing $\{|\rightarrow\rangle, |\uparrow\rangle\}$ as our basis, we start with $\frac{1}{2}$ in state $|\rightarrow\rangle$ and $\frac{1}{2}$ in state $|\uparrow\rangle$, since we are given that the photons are emitted with random polarization. Since the polarization of polaroid $A$ is $|\rightarrow \rangle$, we have that $\frac{1}{2}$ of the photons are allowed to pass through, and are all now in the state $|\rightarrow \rangle$ from being projected onto the polization axis of polaroid $A$. Coming upon polaroid $B$, we have that the polaroid allows $\cos^2$ percent of the photons to pass through that are in state $|\rightarrow \rangle$, and $\sin^2$ percent of the photons that are in state $|\uparrow \rangle$ to pass through (however, none of these reached $B$ because they were filtered out from polaroid $A$). This means that $\frac{1}{2}\cos^2$ percent of the original photons make it through $B$, all of which are now in state $|v\rangle$ after being projected on the polarization axis of $B$. Finally, polaroid $C$ lets through the photons that are in state $|\uparrow \rangle$. Since $\sin^2$ percent of the photons that came through $B$ are in the state $|\uparrow \rangle$, $C$ allows $\sin^2$ percent of the $\frac{1}{2}\cos^2$ that came through $B$ to pass through. This gives us our final result, that $\frac{1}{2}\cos^2\sin^2$ of the original protons hit the screen after $C$.

\paragraph{Problem 2} Do the following pairs of quantum states represent the same state?
\begin{enumerate}[a)] % a), b), c), ...
\item $|0\rangle$ and $-|0\rangle \rightarrow$ Yes, differ by a global phase of $-1$.
\item $|1\rangle$ and $i|1\rangle \rightarrow$ Yes, differ by a global phase of $i$.
\item $\frac{1}{\sqrt{2}}(|0\rangle + |1\rangle)$ and $\frac{1}{\sqrt{2}}(-|0\rangle + i|1\rangle) \rightarrow$ No,there is no common factor that can be pulled out as a global phase
\item $\frac{1}{\sqrt{2}}(|0\rangle + |1\rangle)$ and $\frac{1}{\sqrt{2}}(|0\rangle - |1\rangle) \rightarrow$ No (In fact, this is the Hadamard basis, where $|+\rangle$ and $|-\rangle$ are not the same state)
\item $\frac{1}{\sqrt{2}}(|0\rangle - |1\rangle)$ and $\frac{1}{\sqrt{2}}(|1\rangle - |0\rangle) \rightarrow$ Yes, differ by a global phase of $-1$
\item $\frac{1}{\sqrt{2}}(|0\rangle + i|1\rangle)$ and $\frac{1}{\sqrt{2}}(i|1\rangle - |0\rangle) \rightarrow$ No, there is no common factor that can be used as a global phase
\item $\frac{1}{\sqrt{2}}(|+\rangle + |-\rangle)$ and $|0\rangle \rightarrow$ Yes, differ by a global phase of $1$ (convert to standard basis!)
\item $\frac{1}{\sqrt{2}}(|i\rangle - |-i\rangle)$ and $|1\rangle \rightarrow$ Yes, convert the first statement into the standard basis to see that they differ by a global phase of $i$
\item $\frac{1}{\sqrt{2}}(|i\rangle + |-i\rangle)$ and $\frac{1}{\sqrt{2}}(|-\rangle + |+\rangle) \rightarrow$ Yes, both equal $|0\rangle$ in the standard basis.
\item $\frac{1}{\sqrt{2}}(|0\rangle + e^{i\pi/4}|1\rangle)$ and $\frac{1}{\sqrt{2}}(e^{-i\pi/4}|0\rangle + |1\rangle) \rightarrow$ Yes, they differ by a global phase of $e^{i\pi/4}$
\end{enumerate}

\paragraph{Problem 3 and 4 (Hadamard explicitly stated)} Which of the following states are superpositions with respect to the standard basis, and which are not? For each state that is a superposition, give a basis with respect to which it is not a superposition.
\begin{enumerate}[a)] % a), b), c), ...
\item $|+\rangle \rightarrow$ Superposition wrt. the standard basis, but not wrt. to the Hadamard basis.
\item $\frac{1}{\sqrt{2}}(|+\rangle + |-\rangle) \rightarrow$ This is the same state as $|0\rangle$, so not a superposition wrt. the standard basis. This is a superposition wrt. the Hadamard basis.
\item $\frac{1}{\sqrt{2}}(|+\rangle - |-\rangle) \rightarrow$ This is the same state as $|1\rangle$, so not a superposition wrt. the standard basis. This is a superposition wrt. the Hadamard basis.
\item $\frac{\sqrt{3}}{2}|+\rangle - \frac{1}{2}|-\rangle \rightarrow$ This is a superposition wrt. the standard basis, and is not a superposition wrt. the basis $\{\frac{\sqrt{3}}{2}|+\rangle - \frac{1}{2}|-\rangle,\frac{\sqrt{3}}{2}|+\rangle + \frac{1}{2}|-\rangle\}$ (feel free to correct me if this is wrong). It is also a superposition wrt. the Hadamard basis.
\item $\frac{1}{\sqrt{2}}(|i\rangle - |-i\rangle) \rightarrow$ This is the same state as $i|1\rangle$, so not a superposition wrt. the standard basis. This is a superposition wrt. the Hadamard basis.
\item $\frac{1}{\sqrt{2}}(|0\rangle - |1\rangle) \rightarrow$ This is the same state as $|-\rangle$, so it is a superposition wrt. the standard basis, and is not a superposition wrt. the Hadamard basis.
\end{enumerate}

\paragraph{Problem 5} Give the set of all values of $\theta$ for which the following pairs of states are equivalent.
\begin{enumerate}[a)] % a), b), c), ...
\item $|1\rangle$ and $\frac{1}{\sqrt{2}}(|+\rangle + e^{i\theta}|-\rangle)\rightarrow$ This is true if $e^{i\theta} = -1$, and so $\theta = \{\pi, 3\pi, 5\pi, ...\}$
\item $\frac{1}{\sqrt{2}}(|i\rangle + e^{i\theta}|-i\rangle)$ and $\frac{1}{\sqrt{2}}(|-i\rangle + e^{-i\theta}|i\rangle)\rightarrow$ This is true if $e^{i\theta} = e^{-i\theta} = 1$, and so $\theta = \{0, 2\pi, 4\pi, ...\}$
\item $\frac{1}{2}|0\rangle - \frac{\sqrt{3}}{2}|1\rangle$ and $e^{i\theta}\Big(\frac{1}{2}|0\rangle - \frac{\sqrt{3}}{2}|1\rangle\Big) \rightarrow$ This is true for any number, since $\theta$ just varies the global phase, and the states stay equivalent. Therefore, $\theta = \mathbb{C}$
\end{enumerate}

\paragraph{Problem 6} For each pair and measurement basis, give the possible measurements outcomes and the probability for each.
\begin{enumerate}[a)] % a), b), c), ...
\item $\frac{\sqrt{3}}{2}|0\rangle - \frac{1}{2}|1\rangle$ wrt. $\{|0\rangle,|1\rangle\} \rightarrow$ There is a $\frac{3}{4}$ chance of measuring a $|0\rangle$, and a $\frac{1}{4}$ chance of measuring a $|1\rangle$.
\item $\frac{\sqrt{3}}{2}|1\rangle - \frac{1}{2}|0\rangle$ wrt. $\{|0\rangle,|1\rangle\} \rightarrow$ There is a $\frac{3}{4}$ chance of measuring a $|1\rangle$, and a $\frac{1}{4}$ chance of measuring a $|0\rangle$.
\item $|-i\rangle$ wrt. $\{|0\rangle,|1\rangle\} \rightarrow$ There is a $\frac{1}{2}$ chance of measuring $|0\rangle$, and a $\frac{1}{2}$ chance of measuring a $|1\rangle$.
\item $|0\rangle$ wrt. $\{|+\rangle,|-\rangle\} \rightarrow$ There is a $\frac{1}{2}$ chance of measuring $|+\rangle$, and a $\frac{1}{2}$ chance of measuring a $|-\rangle$ (since $|0\rangle = \frac{1}{\sqrt{2}}(|+\rangle + |-\rangle)$.
\item $\frac{1}{\sqrt{2}}(|0\rangle - |1\rangle)$ wrt. $\{|i\rangle,|-i\rangle\} \rightarrow$ First we convert the state from the $\{|0\rangle,|1\rangle\}$ basis to the $\{|i\rangle,|-i\rangle\}$ basis. This gives us that $\frac{1}{\sqrt{2}}(|0\rangle - |1\rangle) = \frac{1+i}{2}|i\rangle + \frac{1-i}{2}|-i\rangle$. Getting the magnitudes of the vector $\frac{1 \pm i}{2}$, we have that there is an amplitude of $\frac{\sqrt{2}}{2}$ for each component, resulting in a final answer of $\frac{1}{2}$ probability of measuring $|i\rangle$ and a $\frac{1}{2}$ probability of measuring $|-i\rangle$.
\item $|1\rangle$ wrt. $\{|i\rangle,|-i\rangle\} \rightarrow$ First, $|1\rangle$ is equivalent to the state $\frac{-i}{\sqrt{2}}(|i\rangle - |-i\rangle)$, which gives us a result of $\frac{1}{2}$ probability of measuring $|i\rangle$ and a $\frac{1}{2}$ probability of measuring $|-i\rangle$.
\item $|+\rangle$ wrt. $\{\frac{1}{2}|0\rangle + \frac{\sqrt{3}}{2}|1\rangle,\frac{\sqrt{3}}{2}|0\rangle - \frac{1}{2}|1\rangle\} \rightarrow$ This problem is best solved by calculating the projection of the state onto the measurement basis. For the first component of the basis ($\frac{1}{2}|0\rangle + \frac{\sqrt{3}}{2}|1\rangle$), using the dot product, we obtain a vector of magnitude $\frac{1}{2\sqrt{2}} + \frac{\sqrt{3}}{2\sqrt{2}}$ onto that component. For the second component of the basis ($\frac{\sqrt{3}}{2}|0\rangle - \frac{1}{2}|1\rangle$), using the dot product, we obtain a vector of magnitude $\frac{\sqrt{3}}{2\sqrt{2}} - \frac{1}{2\sqrt{2}}$ onto that component. This gives us our final probabilities, a 93.3...\% of measuring $\frac{1}{2}|0\rangle + \frac{\sqrt{3}}{2}|1\rangle$, and 6.6...\% of measuring $\frac{\sqrt{3}}{2}|0\rangle - \frac{1}{2}|1\rangle$.
\end{enumerate}

\paragraph{Problem 7} For each state, describe all orthonormal bases that include that state.
\begin{enumerate}[a)] % a), b), c), ...
\item $\frac{1}{\sqrt{2}}(|0\rangle + i|1\rangle) \rightarrow$ This state is the same state as $|i\rangle$, so an appropriate basis would be $\{|i\rangle,|-i\rangle\}$
\item $\frac{1+i}{2}|0\rangle - \frac{1-i}{2}|1\rangle \rightarrow$
\item $\frac{1}{\sqrt{2}}(|0\rangle + e^{i\pi/6}|1\rangle) \rightarrow$
\item $\frac{1}{2}|+\rangle - \frac{i\sqrt{3}}{2}|-\rangle \rightarrow$
\end{enumerate}

\paragraph{Problem 8} Alice is confused. She understands that $|1\rangle$ and $-|1\rangle$ represent the same state. But she does not understand why that does not imply that $\frac{1}{\sqrt{2}}(|0\rangle + |1\rangle)$ and $\frac{1}{\sqrt{2}}(|0\rangle - |1\rangle)$ would be the same state. Can you help her out?\\
In the first case, the two states differ only by a global phase, and since the global phase has no physical meaning, the states are indistinguishable in nature and are equivalent. However, in the second case, the two states differ by a \textit{relative phase}, meaning that the two states would have different angles when represented in the complex plane, and as such are therefore indistinguishable. This relative phase is what causes $\frac{1}{\sqrt{2}}(|0\rangle + |1\rangle)$ and $\frac{1}{\sqrt{2}}(|0\rangle - |1\rangle)$ to be different states.

\paragraph{Problem 9} In the BB84 protocol, how many bits do Alice and Bob need to compare to have a 90\% chance of detecting Eve's presence?\\
For each bit that Alice and Bob compare, there is a 75\% chance that they do not see a wrong bit. They want to get the probability of seeing these matching bits given Eve eavesdropped to less than 10\%. Therefore, we have $0.75^x<0.10$, where $x$ is the number of bits to compare, leaving a final solution of about $8.00392$ bits needed to detect Eve's presence with 90\% confidence (so in practice, $9$ bits).

\section{Multi-Qubit Systems}

Unlike classical systems, the state space of a quantum system grows exponentially with the number of particles. This is because in a classical system, describing the state of each individual component is enough to describe the system. However, in quantum computing, this is not enough; you must also describe \textit{entangled states}, which cannot be described solely by their individual components. Additionally, there is no known classical metaphor for these entangled states, and the inability to efficiently describe these states on a classical computer is what makes quantum computers so powerful.

\subsection{Quantum State Spaces}

\begin{align*}
\begin{tabular}{ | l | c | }
  \hline			
  System Type & Dimensions of Vector Space \\ \hline
  Classical & $2n$ \\
  Quantum & $2^n$ \\
  \hline  
\end{tabular}
\end{align*}

When in a classical system, the direct sum of two vector spaces $V$ and $W$ is simply the vector space with bases consisting of the union of the bases of the two vector spaces. Therefore,$$dim(V\oplus W)=dim(V)+dim(W)$$
In a more clear explanation, the state of, say, two systems is simply the sum of the states of the two systems. The size of the state space grows linearly with the number of objects.

In a quantum system, we instead take the tensor product of the states of the objects involved in the system, which results in an $nm$ dimensional vector space. All elements of $V\otimes W$ can be written as $$a_1(|\alpha_1\rangle\otimes|\beta_1\rangle)+a_2(|\alpha_2\rangle\otimes|\beta_1\rangle)+...+a_{nm}(|\alpha_n\rangle\otimes|\beta_m\rangle)$$
However, note that most elements of $V \otimes W$ cannot be written as $|v\rangle \otimes |w\rangle$. These states are considered \textit{entangled states}.

\subsection{Entangled States}

\subsection{Chapter Exercises}

\paragraph{Problem 1}

\end{document}
