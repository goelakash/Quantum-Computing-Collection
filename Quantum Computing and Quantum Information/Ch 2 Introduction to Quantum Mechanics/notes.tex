\documentclass[12pt]{article}

\usepackage[T1]{fontenc}
\usepackage[utf8]{inputenc}
\usepackage{graphicx}
\usepackage{xcolor}

\usepackage{tgtermes}

\usepackage[
pdftitle={Introduction to Quantum Mechanics}, 
pdfauthor={Quick Notes by Aaron Vontell, MIT},
colorlinks=true,linkcolor=blue,urlcolor=blue,citecolor=blue,bookmarks=true,
bookmarksopenlevel=2]{hyperref}
\usepackage{amsmath,amssymb,amsthm,textcomp}
\usepackage{enumerate}
\usepackage{multicol}
\usepackage{tikz}

\usepackage{geometry}
\geometry{total={210mm,297mm},
left=25mm,right=25mm,%
bindingoffset=0mm, top=20mm,bottom=20mm}


\linespread{1.3}

\newcommand{\linia}{\rule{\linewidth}{0.5pt}}

% custom theorems if needed
\newtheoremstyle{mytheor}
    {1ex}{1ex}{\normalfont}{0pt}{\scshape}{.}{1ex}
    {{\thmname{#1 }}{\thmnumber{#2}}{\thmnote{ (#3)}}}

\theoremstyle{mytheor}
\newtheorem{defi}{Definition}

% my own titles
\makeatletter
\renewcommand{\maketitle}{
\begin{center}
\vspace{2ex}
{\huge \textsc{\@title}}
\vspace{1ex}
\\
by Michael A. Nielsen and Isaac L. Chuang\\
\linia\\
\@author \hfill \@date
\vspace{4ex}
\end{center}
}
\makeatother
%%%

% custom footers and headers
\usepackage{fancyhdr,lastpage}
\pagestyle{fancy}
\lhead{}
\chead{}
\rhead{}
\lfoot{}
\cfoot{}
\rfoot{Page \thepage\ /\ \pageref*{LastPage}}
\renewcommand{\headrulewidth}{0pt}
\renewcommand{\footrulewidth}{0pt}
%

%%%----------%%%----------%%%----------%%%----------%%%

\begin{document}

\title{Introduction to Quantum Mechanics (from QC and QI)}

\author{Quick Notes by Aaron Vontell, MIT EECS}

\date{08/31/2016}

\maketitle

\section{Linear Algebra}

\subsection{Overview of Dirac Notation}

\begin{align*}
\begin{tabular}{ | c | l | }
  \hline			
  Notation & Description \\ \hline
  $z^*$ & Complex conjugate of the complex number $z$ \\
  $|\psi\rangle$ & Column vector, also known as a $ket$\\
  $\langle\psi|$ & Row vector, also known as a $bra$\\
  $\langle\varphi|\psi\rangle$ & Inner product between vectors $|\varphi\rangle$ and $|\psi\rangle$\\
  $|\varphi\rangle\otimes|\psi\rangle$ & Tensor product of $|\varphi\rangle$ and $|\psi\rangle$\\
  $|\varphi\rangle|\psi\rangle$ & Tensor product of $|\varphi\rangle$ and $|\psi\rangle$\\
  $A^*$ & Complex conjugate of the $A$ matrix \\
  $A^T$ & Transposed of the $A$ matrix \\
  $A$\textdagger & Hermitian conjugate or adjoint of the $A$ matrix $= (A^T)^*$ \\
  \hline  
\end{tabular}
\end{align*}

\subsection{Bases and Linear Independence}

\paragraph{Exercise 2.1 Linear Dependence} Show that $(1,-1),(1,2)$ and $(2,1)$ are linearly dependent.\\
This is equivalent to finding the set of complex numbers $a_1,...,a_n$ that satisfy the equation $a_1|v_1\rangle + a_2|v_2\rangle + ... + a_n|v_n\rangle = 0$. Therefore, we have
$$a_1(1,-1) + a_2(1,2) + a_3(2,1) = 0$$
One set that satisfies this equality (while maintaining $a_i \ne 0$ for at least one value of $i$) is $$a_1 = 1, a_2 = 1, and a_3 = -1$$ Therefore, this set of vectors is linearly dependent.

\subsection{Linear Operators and Matrices}

An $m$ by $n$ matrix $A$ with entries $A_{ij}$ is a linear operator sending vectors in the vector space $\mathbb{C}^n$  to the vector space $\mathbb{C}^m$. The operation can be written as (for all $i$) $$A\left(\Sigma a_i|v_i\rangle\right) = \Sigma a_iA|v_i\rangle$$

Four important matrices are the \textbf{Pauli matrices}:
  $$\sigma_0=\sigma_I=I=
  \begin{bmatrix}
    1 & 0\\
    0 & 1
  \end{bmatrix}\qquad
  \sigma_1=\sigma_x=X=
  \begin{bmatrix}
    0 & 1\\
    1 & 0
  \end{bmatrix}$$
  $$\sigma_2=\sigma_y=Y=
  \begin{bmatrix}
    0 & -i\\
    i & 0
  \end{bmatrix}\qquad
  \sigma_3=\sigma_z=Z=
  \begin{bmatrix}
    1 & 0\\
    0 & -1
  \end{bmatrix}$$
  
\subsection{Inner Products}

$$((y_1,...,y_n),(z_1,...,z_n)) = \Sigma y_i^*z_i$$

Two vectors $|v\rangle$ and $|w\rangle$ are orthogonal if $\langle v|w\rangle=0$

Also note that in the context of this material, an inner product space is exactly the same thing as a Hilbert space.

The norm of a vector $|v\rangle$ is given by $||\ |v\rangle|| = \sqrt{\langle v|v\rangle}$, where a norm of 1 indicates that $v$ is a unit vector. If not a unit vector, we can normalize $|v\rangle$ by dividing it by its norm.

\end{document}
